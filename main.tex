\documentclass{beamer}
\usetheme{Berlin}

% Define custom colors
\definecolor{navy}{RGB}{0,0,128}
\definecolor{lightblue}{RGB}{173,216,230}
\definecolor{white}{RGB}{255,255,255}

% Apply the colors to various elements
\setbeamercolor{structure}{fg=navy}
\setbeamercolor{palette primary}{bg=navy, fg=white}
\setbeamercolor{palette secondary}{bg=lightblue, fg=black}
\setbeamercolor{palette tertiary}{bg=navy, fg=white}
\setbeamercolor{palette quaternary}{bg=lightblue, fg=black}
\setbeamercolor{frametitle}{bg=navy, fg=white}
\setbeamercolor{title}{fg=navy, bg=white} % Ensure title is visible

\title{ESG KPIs for Monitoring Water Usage}
\author{Katsiaryna Bahamazava, PhD.}
\institute{Mathematical Department of Politecnico di Torino}
\date{\today}

\begin{document}

\frame{\titlepage}

\section{Introduction}

\begin{frame}
\frametitle{Introduction}
\begin{itemize}
    \item ESG (Environmental, Social, and Governance) KPIs help companies measure their performance in sustainable and ethical practices.
    \item This presentation focuses on water usage as a critical environmental KPI.
    \item We will demonstrate how AI can be utilized to predict water usage and improve sustainability efforts.
\end{itemize}
\end{frame}

\section{Environmental KPIs}

\begin{frame}
\frametitle{Environmental KPIs}
\begin{itemize}
    \item \textbf{Water Usage}
        \begin{itemize}
            \item Measures the total amount of water used by the company.
            \item Example: Track water usage in cubic meters annually and implement strategies to reduce consumption.
        \end{itemize}
    \item \textbf{GHG Emissions}
        \begin{itemize}
            \item Measures the total greenhouse gases emitted by the company.
            \item Example: Reduce GHG emissions by 25\% by 2025.
        \end{itemize}
    \item \textbf{Energy Consumption}
        \begin{itemize}
            \item Tracks the total energy used by the company from all sources (electricity, gas, renewable energy, etc.).
            \item Example: Report total energy consumption in MWh annually.
        \end{itemize}
    \item \textbf{Waste Generated}
        \begin{itemize}
            \item Measures the total amount of waste produced.
            \item Example: Report total waste generated in metric tons annually.
        \end{itemize}
\end{itemize}
\end{frame}

\section{Predicting Water Usage with AI}

\begin{frame}
\frametitle{Predicting Water Usage with AI}
\begin{itemize}
    \item AI can be used to predict water usage, enabling proactive management and resource optimization.
    \item Key steps in the prediction process:
\end{itemize}
\end{frame}

\begin{frame}
\frametitle{Step 1: Data Collection}
\begin{itemize}
    \item Collect relevant data for water usage prediction:
        \begin{itemize}
            \item Historical water usage data.
            \item Environmental data (temperature, humidity).
            \item Production data (production levels, operational hours).
            \item Workforce data (number of employees, shifts).
        \end{itemize}
\end{itemize}
\end{frame}

\begin{frame}
\frametitle{Step 2: Data Preprocessing}
\begin{itemize}
    \item Clean and preprocess the data:
        \begin{itemize}
            \item Handle missing values.
            \item Encode categorical variables.
            \item Scale numerical features.
        \end{itemize}
\end{itemize}
\end{frame}

\begin{frame}
\frametitle{Step 3: Feature Engineering}
\begin{itemize}
    \item Create new features to improve the model's accuracy:
        \begin{itemize}
            \item Seasonal components (e.g., monthly averages).
            \item Lag features (e.g., water usage from previous weeks).
            \item Interaction terms (e.g., interaction between temperature and production level).
        \end{itemize}
\end{itemize}
\end{frame}

\begin{frame}
\frametitle{Step 4: Model Training}
\begin{itemize}
    \item Train a neural network model to predict water usage:
        \begin{itemize}
            \item Define the neural network architecture.
            \item Compile the model with appropriate loss function and optimizer.
            \item Train the model on the training dataset.
        \end{itemize}
\end{itemize}
\end{frame}

\begin{frame}
\frametitle{Step 5: Model Evaluation}
\begin{itemize}
    \item Evaluate the model using test data:
        \begin{itemize}
            \item Calculate metrics such as Mean Squared Error (MSE), Mean Absolute Error (MAE), and R-squared.
            \item Plot training history to observe performance over epochs.
        \end{itemize}
\end{itemize}
\end{frame}

\begin{frame}
\frametitle{Step 6: Prediction and Scenario Analysis}
\begin{itemize}
    \item Use the trained model to predict water usage under various scenarios:
        \begin{itemize}
            \item Predict future water usage based on planned production increases.
            \item Assess the impact of environmental changes on water usage.
        \end{itemize}
    \item Implement these predictions into the company's resource management strategy.
\end{itemize}
\end{frame}

\section{Conclusion}

\begin{frame}
\frametitle{Conclusion}
\begin{itemize}
    \item Monitoring and predicting water usage is crucial for sustainable resource management.
    \item AI-powered predictions allow for proactive decision-making and improved efficiency.
    \item Regularly updating and refining the model ensures it remains accurate and valuable.
\end{itemize}
\end{frame}

\section{License}

\begin{frame}
\frametitle{License}
\begin{itemize}
    \item This work is licensed under a Creative Commons Attribution-NonCommercial-ShareAlike 4.0 International License.
    \item You must give appropriate credit, provide a link to the license, and indicate if changes were made. You may do so in any reasonable manner, but not in any way that suggests the licensor endorses you or your use.
    \item You may not use the material for commercial purposes.
    \item If you remix, transform, or build upon the material, you must distribute your contributions under the same license as the original.
\end{itemize}
\includegraphics[width=1.5cm]{heart.red.png}
\end{frame}

\end{document}
