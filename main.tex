\documentclass{beamer}
\usetheme{Berlin}

% Define custom colors
\definecolor{navy}{RGB}{0,0,128}
\definecolor{lightblue}{RGB}{173,216,230}
\definecolor{white}{RGB}{255,255,255}

% Apply the colors to various elements
\setbeamercolor{structure}{fg=navy}
\setbeamercolor{palette primary}{bg=navy, fg=white}
\setbeamercolor{palette secondary}{bg=lightblue, fg=black}
\setbeamercolor{palette tertiary}{bg=navy, fg=white}
\setbeamercolor{palette quaternary}{bg=lightblue, fg=black}
\setbeamercolor{frametitle}{bg=navy, fg=white}
\setbeamercolor{title}{fg=navy, bg=white} % Ensure title is visible

\title{ESG KPIs for Monitoring Water Usage and GHG Emissions}
\author{Katsiaryna Bahamazava, PhD.}
\institute{Mathematical Department of Politecnico di Torino}
\date{\today}

\begin{document}

\frame{\titlepage}

\section{Introduction}

\section{Introduction}

\begin{frame}
\frametitle{Introduction}
\begin{itemize}
    \item ESG (Environmental, Social, and Governance) KPIs help companies measure their performance in sustainable and ethical practices.
    \item This presentation covers examples of KPIs for Environmental, Social, and Governance aspects.
    \item We will also demonstrate how AI can predict KPIs using the water usage as an example.
\end{itemize}
\end{frame}

\section{Environmental KPIs}

\begin{frame}
\frametitle{Environmental KPIs}
\begin{itemize}
    \item \textbf{GHG Emissions}
        \begin{itemize}
            \item Measures the total greenhouse gases emitted by the company.
            \item Example: Reduce GHG emissions by 25\% by 2025.
        \end{itemize}
    \item \textbf{Energy Consumption}
        \begin{itemize}
            \item Tracks the total energy used by the company from all sources (electricity, gas, renewable energy, etc.).
            \item Example: Report total energy consumption in MWh annually.
        \end{itemize}
    \item \textbf{Water Consumption}
        \begin{itemize}
            \item Measures the total amount of water used by the company.
            \item Example: Report total water usage in cubic meters annually.
        \end{itemize}
    \item \textbf{Waste Generated}
        \begin{itemize}
            \item Measures the total amount of waste produced.
            \item Example: Report total waste generated in metric tons annually.
        \end{itemize}
\end{itemize}
\end{frame}

\section{Social KPIs}

\begin{frame}
\frametitle{Social KPIs}
\begin{itemize}
    \item \textbf{Turnover Rates}
        \begin{itemize}
            \item Measures the rate at which employees leave the company.
            \item Example: Reduce turnover rates by 15\% by 2023.
        \end{itemize}
    \item \textbf{Percentage of Diverse Employees}
        \begin{itemize}
            \item Tracks diversity in the workforce.
            \item Example: Increase percentage of diverse employees by 20\% by 2024.
        \end{itemize}
    \item \textbf{Pay Equity Ratios}
        \begin{itemize}
            \item Measures the pay gap between different groups.
            \item Example: Achieve pay equity by 2025.
        \end{itemize}
    \item \textbf{Number of Product Recalls}
        \begin{itemize}
            \item Measures the number of product recalls due to safety issues.
            \item Example: Maintain zero product recalls annually.
        \end{itemize}
\end{itemize}
\end{frame}

\section{Governance KPIs}

\begin{frame}
\frametitle{Governance KPIs}
\begin{itemize}
    \item \textbf{Board Diversity}
        \begin{itemize}
            \item Measures diversity within the company's board.
            \item Example: Achieve 40\% board diversity by 2023.
        \end{itemize}
    \item \textbf{Number of Ethical Training Hours}
        \begin{itemize}
            \item Measures the total hours of ethical training completed by employees.
            \item Example: Increase ethical training hours by 30\% by 2024.
        \end{itemize}
    \item \textbf{Number of Data Breaches}
        \begin{itemize}
            \item Measures the number of data breaches the company experiences.
            \item Example: Maintain zero data breaches annually.
        \end{itemize}
    \item \textbf{Response Time to Data Breaches}
        \begin{itemize}
            \item Measures the time taken to respond to data breaches.
            \item Example: Reduce average response time to data breaches to under 24 hours.
        \end{itemize}
\end{itemize}
\end{frame}

\section{Predicting Water Usage with AI}

\begin{frame}
\frametitle{Predicting Water Usage with AI}
\begin{itemize}
    \item AI can be used to predict water usage, enabling proactive management and resource optimization.
    \item Key steps in the prediction process:
\end{itemize}
\end{frame}

\begin{frame}
\frametitle{Step 1: Data Collection}
\begin{itemize}
    \item Collect relevant data for water usage prediction:
        \begin{itemize}
            \item Historical water usage data.
            \item Environmental data (temperature, humidity).
            \item Production data (production levels, operational hours).
            \item Workforce data (number of employees, shifts).
        \end{itemize}
\end{itemize}
\end{frame}

\begin{frame}
\frametitle{Step 2: Data Preprocessing}
\begin{itemize}
    \item Clean and preprocess the data:
        \begin{itemize}
            \item Handle missing values.
            \item Encode categorical variables.
            \item Scale numerical features.
        \end{itemize}
\end{itemize}
\end{frame}

\begin{frame}
\frametitle{Step 3: Feature Engineering}
\begin{itemize}
    \item Create new features to improve the model's accuracy:
        \begin{itemize}
            \item Seasonal components (e.g., monthly averages).
            \item Lag features (e.g., water usage from previous weeks).
            \item Interaction terms (e.g., interaction between temperature and production level).
        \end{itemize}
\end{itemize}
\end{frame}

\begin{frame}
\frametitle{Step 4: Model Training}
\begin{itemize}
    \item Train a neural network model to predict water usage:
        \begin{itemize}
            \item Define the neural network architecture.
            \item Compile the model with appropriate loss function and optimizer.
            \item Train the model on the training dataset.
        \end{itemize}
\end{itemize}
\end{frame}

\begin{frame}
\frametitle{Step 5: Model Evaluation}
\begin{itemize}
    \item Evaluate the model using test data:
        \begin{itemize}
            \item Calculate metrics such as Mean Squared Error (MSE), Mean Absolute Error (MAE), and R-squared.
            \item Plot training history to observe performance over epochs.
        \end{itemize}
\end{itemize}
\end{frame}

\begin{frame}
\frametitle{Step 6: Prediction and Scenario Analysis}
\begin{itemize}
    \item Use the trained model to predict water usage under various scenarios:
        \begin{itemize}
            \item Predict future water usage based on planned production increases.
            \item Assess the impact of environmental changes on water usage.
        \end{itemize}
    \item Implement these predictions into the company's resource management strategy.
\end{itemize}
\end{frame}

\section{Transition to GHG Emissions}

\begin{frame}
\frametitle{Expanding Focus: From Water Usage to GHG Emissions}
\begin{itemize}
    \item While monitoring water usage is critical, another key ESG KPI is GHG emissions.
    \item Reducing GHG emissions is essential for aligning with global sustainability goals like the Paris Agreement.
    \item Next, we will explore a group activity focused on developing strategies to monitor and reduce GHG emissions.
\end{itemize}
\end{frame}

\section{Conclusion}



\section{Group Activity}

\begin{frame}
\frametitle{Group Activity: Developing a Strategy to Monitor and Reduce GHG Emissions}
\begin{itemize}
    \item Objective: Discuss and develop a strategic plan for monitoring and reducing GHG emissions as part of a company's ESG initiatives.
    \item Focus: Identify key emission sources, relevant KPIs, data collection methods, and potential reduction strategies.
    \item Outcome: Each group will present their strategy, highlighting challenges and solutions.
\end{itemize}
\end{frame}

\begin{frame}
\frametitle{Scenario Overview}
\begin{itemize}
    \item You are part of the sustainability team at an international manufacturing company.
    \item The company aims to reduce its GHG emissions by 40\% over the next decade in line with the Paris Agreement.
    \item The challenge: Develop a comprehensive system to monitor and manage GHG emissions across multiple regions and operations.
\end{itemize}
\end{frame}

\begin{frame}
\frametitle{Discussion Points for Your Group}
\begin{itemize}
    \item \textbf{Identify Key Emission Sources:} Where do most of the GHG emissions come from in your company?
    \item \textbf{Choose Relevant KPIs:} What KPIs should be tracked? 
    \item \textbf{Data Collection:} What data do you need? How will you gather and integrate it across different sites?
    \item \textbf{Reduction Strategies:} How can the company reduce emissions? 
    \item \textbf{Challenges and Solutions:} What challenges might you face? How will you overcome them?
\end{itemize}
\end{frame}

\begin{frame}
\frametitle{Group Presentation Instructions}
\begin{itemize}
    \item Each group will have 5-7 minutes to present their strategy.
    \item Focus on:
    \begin{itemize}
        \item Key emission sources and chosen KPIs.
        \item Data collection methods and challenges.
        \item Proposed strategies for reducing GHG emissions.
        \item Anticipated challenges and solutions.
    \end{itemize}
    \item Be prepared to answer questions and discuss your approach with the class.
\end{itemize}
\end{frame}

\begin{frame}
\frametitle{Class Discussion}
\begin{itemize}
    \item After all groups have presented, we will have an open discussion.
    \item Topics to explore:
    \begin{itemize}
        \item Differences and similarities in group strategies.
        \item Feasibility and practicality of proposed solutions.
        \item New ideas or perspectives that emerged during the presentations.
    \end{itemize}
\end{itemize}
\end{frame}

\begin{frame}
\frametitle{Summary and Key Takeaways}
\begin{itemize}
    \item Monitoring ESG KPIs is crucial for sustainability and ESG compliance.
    \item Data-driven strategies can significantly enhance the effectiveness of these initiatives.
    \item Collaboration and strategic planning are key to overcoming challenges in ESG implementation.
    \item Continuous improvement and adaptation are necessary to meet long-term sustainability goals.
  \item AI-powered predictions allow for proactive decision-making and improved efficiency.
\end{itemize}
\end{frame}

\section{License}

\begin{frame}
\frametitle{License}
\begin{itemize}
    \item This work is licensed under a Creative Commons Attribution-NonCommercial-ShareAlike 4.0 International License.
    \item You must give appropriate credit, provide a link to the license, and indicate if changes were made. You may do so in any reasonable manner, but not in any way that suggests the licensor endorses you or your use.
    \item You may not use the material for commercial purposes.
    \item If you remix, transform, or build upon the material, you must distribute your contributions under the same license as the original.
\end{itemize}
\includegraphics[width=1.5cm]{heart.red.png}

\end{frame}
\end{document}
